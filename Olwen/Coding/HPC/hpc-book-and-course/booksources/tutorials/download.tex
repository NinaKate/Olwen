%%%%%%%%%%%%%%%%%%%%%%%%%%%%%%%%%%%%%%%%%%%%%%%%%%%%%%%%%%%%%%%%
%%%%%%%%%%%%%%%%%%%%%%%%%%%%%%%%%%%%%%%%%%%%%%%%%%%%%%%%%%%%%%%%
%%%%
%%%% This text file is part of the source of 
%%%% `Introduction to High-Performance Scientific Computing'
%%%% by Victor Eijkhout, copyright 2012-7
%%%%
%%%% This book is distributed under a Creative Commons Attribution 3.0
%%%% Unported (CC BY 3.0) license and made possible by funding from
%%%% The Saylor Foundation \url{http://www.saylor.org}.
%%%%
%%%%
%%%%%%%%%%%%%%%%%%%%%%%%%%%%%%%%%%%%%%%%%%%%%%%%%%%%%%%%%%%%%%%%
%%%%%%%%%%%%%%%%%%%%%%%%%%%%%%%%%%%%%%%%%%%%%%%%%%%%%%%%%%%%%%%%

In the theory part of this book you learned mathematical models can be
translated to algorithms that can be realized efficiently on modern
hardware. You learned how data structures and coding decisions
influence the performance of your code. In other words, you should now
have all the tools to write programs that solve scientific problems.

This would be all you would need to know,
if there was any guarantee that a correctly derived algorithm and
well designed data structure could immediately be turned into a
correct program.
Unfortunately, there is more to programming than that.
You need some tools to be an effective scientific
programmer.

You can find a downloadable collection of tutorials to complement this
book
at \url{https://bitbucket.org/VictorEijkhout/hpc-book-and-course/downloads/}.
