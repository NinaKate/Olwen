% -*- latex -*-
%%%%%%%%%%%%%%%%%%%%%%%%%%%%%%%%%%%%%%%%%%%%%%%%%%%%%%%%%%%%%%%%
%%%%%%%%%%%%%%%%%%%%%%%%%%%%%%%%%%%%%%%%%%%%%%%%%%%%%%%%%%%%%%%%
%%%%
%%%% This text file is part of the source of 
%%%% `Parallel Computing'
%%%% by Victor Eijkhout, copyright 2012-6
%%%%
%%%% .tex : 
%%%%
%%%%%%%%%%%%%%%%%%%%%%%%%%%%%%%%%%%%%%%%%%%%%%%%%%%%%%%%%%%%%%%%
%%%%%%%%%%%%%%%%%%%%%%%%%%%%%%%%%%%%%%%%%%%%%%%%%%%%%%%%%%%%%%%%

N-body problems describe the motion of particles under the influence
of forces such as gravity. There are many approaches to this problem,
some exact, some approximate. Here we will explore a number of them.

For background reading see \HPSCref{app:discrete}.

\Level 0 {Solution methods}

It is not in the scope of this course to give a systematic treatment
of all methods for solving the N-body problem, whether exactly or
approximately, so we will just consider a representative selection.

\begin{enumerate}
\item Full $N^2$ methods. These compute all interactions, which is the
  most accurate strategy, but also the most computationally demanding.
\item Cutoff-based methods. These use the basic idea of the
  $N^2$ interactions, but reduce the complexity by imposing a cutoff
  on the interaction distance.
\item Tree-based methods. These apply a coarsening scheme to distant
  interactions to lower the computational complexity.
\end{enumerate}

\Level 0 {Shared memory approaches}

\Level 0 {Distributed memory approaches}
