% -*- latex -*-
%%%%%%%%%%%%%%%%%%%%%%%%%%%%%%%%%%%%%%%%%%%%%%%%%%%%%%%%%%%%%%%%
%%%%%%%%%%%%%%%%%%%%%%%%%%%%%%%%%%%%%%%%%%%%%%%%%%%%%%%%%%%%%%%%
%%%%
%%%% This text file is part of the source of 
%%%% `Parallel Programming in MPI and OpenMP'
%%%% by Victor Eijkhout, copyright 2012-6
%%%%
%%%% mpi-commbasic.tex : communicator basics
%%%%
%%%%%%%%%%%%%%%%%%%%%%%%%%%%%%%%%%%%%%%%%%%%%%%%%%%%%%%%%%%%%%%%
%%%%%%%%%%%%%%%%%%%%%%%%%%%%%%%%%%%%%%%%%%%%%%%%%%%%%%%%%%%%%%%%

\Level 0 {Communicator basics}

A communicator is an object describing a group of processes. In many 
applications all processes work together closely coupled, and the
only communicator you need is \n{MPI_COMM_WORLD}. However, there are 
circumstances where you want one subset of processes to operate 
independently of another subset. For example:
\begin{itemize}
\item If processors are organized in a $2\times2$ grid, you may want
  to do broadcasts inside a row or column. 
\item For an application that includes a producer and a consumer part,
  it makes sense to split the processors accordingly.
\end{itemize}
In this section we will see mechanisms for defining new communicators
and sending messages between communicators.

An important reason for using communicators is the development of
software libraries. If the routines in a library use their own communicator
(even if it is a duplicate of the `outside' communicator), there
will never be a confusion between message tags inside and outside the 
library.

There are three predefined communicators:
\begin{itemize}
\item \indexmpishow{MPI_COMM_WORLD} comprises all processes that were started 
  together by \indexterm{mpirun} (or some related program).
\item \indexmpishow{MPI_COMM_SELF} is the communicator that contains only
   the current process.
\item \indexmpishow{MPI_COMM_NULL} is the invalid communicator. Routines
  that construct communicators can give this as result if an error occurs.
\end{itemize}
%Implementationally, communicators are integers, so you can use a 
%simple test for equality.

In some applications you will find yourself regularly creating new
communicators, using the mechanisms described below. In that case, you
should de-allocate communicators with \indexmpishow{MPI_Comm_free} when
you're done with them.

