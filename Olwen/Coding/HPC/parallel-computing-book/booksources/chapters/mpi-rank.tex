% -*- latex -*-
%%%%%%%%%%%%%%%%%%%%%%%%%%%%%%%%%%%%%%%%%%%%%%%%%%%%%%%%%%%%%%%%
%%%%%%%%%%%%%%%%%%%%%%%%%%%%%%%%%%%%%%%%%%%%%%%%%%%%%%%%%%%%%%%%
%%%%
%%%% This text file is part of the source of 
%%%% `Parallel Programming in MPI and OpenMP'
%%%% by Victor Eijkhout, copyright 2012-6
%%%%
%%%% mpi-rank.tex : comm size and rank and such
%%%%
%%%%%%%%%%%%%%%%%%%%%%%%%%%%%%%%%%%%%%%%%%%%%%%%%%%%%%%%%%%%%%%%
%%%%%%%%%%%%%%%%%%%%%%%%%%%%%%%%%%%%%%%%%%%%%%%%%%%%%%%%%%%%%%%%

\Level 0 {Processor identification}
\label{sec:rank-size}

Since all processes in an MPI job are instantiations of the same executable,
you'd think that they all execute the exact same instructions,
which would not be terribly useful.
To distinguish between processors, MPI provides two calls
\begin{enumerate}
\item \indexmpishow{MPI_Comm_size} reports how many processes there are in all; and
\item \indexmpishow{MPI_Comm_rank} states what the number of the process is.
\end{enumerate}
In other words, each process can find out `I~am process~5
out of a total of~20'.

\mpiRoutineRef{MPI_Comm_size}
%
and
%
\mpiRoutineRef{MPI_Comm_rank}

\begin{exercise}
  \label{ex:hello3}
  Write a program where each process prints out message
  reporting its number, and how many processes there are.

  Write a second version of this program, where each process opens a
  unique file and writes to it. \emph{On some clusters this may not be
    advisable if you have large numbers of processors, since it can
    overload the file system.}
\end{exercise}

\begin{exercise}
  \label{ex:hello4}
  Write a program where only the process with number zero
  reports on how many processes there are in total.
\end{exercise}

\endinput
This is probably about the simplest MPI program:
\verbatimsnippet{hello}
%
\verbatimsnippet{hellop}

