%
% This is a comment. It does not appear in the final document. All comments
% are on lines that start with a percent sign.
%
%
% All documents start with a \documentclass command. This one is to user
% the revtex4 macro package from the APS, used for Physical Review.
\documentclass[preprint,aps]{revtex4}
\usepackage[utf8]{inputenc}
\usepackage{amsmath}
\usepackage{amsfonts}
\usepackage{amssymb}
\usepackage{graphicx}
\usepackage{placeins}
\usepackage{tensor}
%
% This loads a package for displaying figures. An example of how to display
% a figure is below.
%
\usepackage{graphicx}

%
% All documents start with this command in LaTeX.
%
\begin{document}
Alright, this REALLY ought to be very straightforward, I have no idea why I'm having such trouble, what the deuce. Anyway, this is me walking you through my math:
Start with $a_{eq}$. I'm following the procedure shown here[\url{http://www.tapir.caltech.edu/~chirata/ph217/lec06.pdf}] (and using their values for the observables, like $\Omega_mh^2$). The method is as follows: radiation energy density goes as $\rho_r \propto a^{-4}$, while matter energy density goes as $\rho_m \propto a^{-3}$. At matter-radiation equality, $\rho_m=\rho_r$. If you take the ratio of these, we get 
\begin{equation}
\frac{\rho_m}{\rho_r}\propto\frac{a^{-3}}{a^{-4}}=a.
\end{equation}
That means that I can just take the ratio of the current energy densities, 
\begin{equation}
a_{eq}=\frac{\rho_{r0}}{\rho_{m0}}.
\end{equation}
The current radiation energy density is, according to that page, $7.8*10^{-34}g/cm^3$. The current matter density is $0.240512*10^{-29}g/cm^3$. If you take that ratio, you get $3.51563*10^{-4}$. This agrees with estimates I find elsewhere, using similar methods (I haven't seen anything in particular using any other method, though due to my internet problems, that might not mean much.) The problem is: We have just asserted and used the fact that $\rho_r=\rho_{r0}a^{-4}$, and if I go and plug in that value of $a_{eq}$ to get the value of radiation energy density at equality, I find that the radiation density alone at equality was $5.10604*10^{-20}g/cm^3$, significantly higher than the critical density for flatness for the universe. \textbf{DANGER, DANGER WILL ROBINSON, SOMETHING HAS GONE TERRIBLY WRONG.} I cannot manage to come up with another way to calculate this, though that might simply be because I can't find any formulae for $a$ that will permit it, curse my lack of textbooks and internet. 

If, however, I continue with this value for $a_{eq}$, I continue on to my equation for $H_{eq}$, namely: 
\begin{equation}
H_{eq}^2=2\Omega_m H_0^2 a_{eq}^{-3}
\end{equation}
...which brings me to another question: I am not sure what value of $H_0$ I should be using, or what units I should be aiming for. That's rather a side issue, however. If I use some reasonable estimates of $\Omega_m=0.3$ and $H_0=70$ as ballparks, I find that $H_{eq}=8.22564*10^6$. This seems almost farcically large to me, but I have not come up with a way to definitely prove it to be inaccurate, so it might just be that this is a number much, much bigger than I was expecting. 

The biggest problem here is how I am to calculate $N_{RD}$. This is the number of e-folds of expansion during the radiation-dominated era, that is, between nucleosynthesis and radiation-matter equality. This one has been stumping me a bit, and it was in trying to figure out how to calculate this one that I realized that what I had wasn't working. My first attempt was to calculate it as $N_{RD}=\log[a_{eq}/a_{nuc}]$, with $a_{nuc}$ the value of $a$ at nucleosynthesis, but I don't know how to calculate $a_{nuc}$. So then I figured that, well, actually all I needed was the amount of expansion, and nucleosynthesis was at $200-300$ seconds, while equality was at $~10^{11}$ seconds, so just the square root of the amount of time elapsed would do it, right? Only I don't think it makes much sense to model this as purely radiation, surely I ought to actually take the changing dynamics of the matter-radiation ratio into account, that would alter the growth rate, it isn't a step function, and moreover this isn't a slow roll scenario, so presumably what I actually ought to be doing is$N_{RD}=\log[a_{eq}H_{eq}/a_{nuc}H_{nuc}]$, and I tried, but I just got increasingly farcical answers (negative e-foldings, for example). So this part in particular confuses me because unlike the others where I have something that OUGHT to be working but apparently isn't, I can't even figure out how I should be approaching this. 

The last thing is, it occurs to me that I might not actually need/want to use $N_k-\ln{T_{re}} = 68$, but rather just go with "And I know the temperature at the end of reheating [since I'm taking that to be nucleosynthesis]" and set $T_{re}$ equal to that known value (which appears to be 0.05 MeV)?
Anyway, I've written up the code to make the calculations given user inputs of the various things that shouldn't vary between potentials ($H_0$, $N_{RD}$, etc.) and tried both approaches to get $N_{min}$, and I get egregriously wrong results both ways (I do not believe that $N_{min}$ should be either over 1000 or negative), but I honestly can't tell if that's a problem with the values I'm putting into it or what I'm doing with them. (please find code attached as well).  


\section{Numbers}
Alright, so if I do my calculations carefully, in natural units, I get the following values for components of the $N_{re}$ equation
\end{document}

